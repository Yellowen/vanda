% Generated by Sphinx.
\documentclass[letterpaper,10pt,english]{manual}
\usepackage[utf8]{inputenc}
\usepackage[T1]{fontenc}
\usepackage{babel}
\usepackage{times}
\usepackage[Bjarne]{fncychap}
\usepackage{longtable}
\usepackage{sphinx}


\title{Dina Documentation}
\date{August 26, 2010}
\release{0.2}
\author{Dina developers community}
\newcommand{\sphinxlogo}{}
\renewcommand{\releasename}{Release}
\makeindex
\makemodindex

\makeatletter
\def\PYG@reset{\let\PYG@it=\relax \let\PYG@bf=\relax%
    \let\PYG@ul=\relax \let\PYG@tc=\relax%
    \let\PYG@bc=\relax \let\PYG@ff=\relax}
\def\PYG@tok#1{\csname PYG@tok@#1\endcsname}
\def\PYG@toks#1+{\ifx\relax#1\empty\else%
    \PYG@tok{#1}\expandafter\PYG@toks\fi}
\def\PYG@do#1{\PYG@bc{\PYG@tc{\PYG@ul{%
    \PYG@it{\PYG@bf{\PYG@ff{#1}}}}}}}
\def\PYG#1#2{\PYG@reset\PYG@toks#1+\relax+\PYG@do{#2}}

\def\PYG@tok@gd{\def\PYG@tc##1{\textcolor[rgb]{0.63,0.00,0.00}{##1}}}
\def\PYG@tok@gu{\let\PYG@bf=\textbf\def\PYG@tc##1{\textcolor[rgb]{0.50,0.00,0.50}{##1}}}
\def\PYG@tok@gt{\def\PYG@tc##1{\textcolor[rgb]{0.00,0.25,0.82}{##1}}}
\def\PYG@tok@gs{\let\PYG@bf=\textbf}
\def\PYG@tok@gr{\def\PYG@tc##1{\textcolor[rgb]{1.00,0.00,0.00}{##1}}}
\def\PYG@tok@cm{\let\PYG@it=\textit\def\PYG@tc##1{\textcolor[rgb]{0.25,0.50,0.56}{##1}}}
\def\PYG@tok@vg{\def\PYG@tc##1{\textcolor[rgb]{0.73,0.38,0.84}{##1}}}
\def\PYG@tok@m{\def\PYG@tc##1{\textcolor[rgb]{0.13,0.50,0.31}{##1}}}
\def\PYG@tok@mh{\def\PYG@tc##1{\textcolor[rgb]{0.13,0.50,0.31}{##1}}}
\def\PYG@tok@cs{\def\PYG@tc##1{\textcolor[rgb]{0.25,0.50,0.56}{##1}}\def\PYG@bc##1{\colorbox[rgb]{1.00,0.94,0.94}{##1}}}
\def\PYG@tok@ge{\let\PYG@it=\textit}
\def\PYG@tok@vc{\def\PYG@tc##1{\textcolor[rgb]{0.73,0.38,0.84}{##1}}}
\def\PYG@tok@il{\def\PYG@tc##1{\textcolor[rgb]{0.13,0.50,0.31}{##1}}}
\def\PYG@tok@go{\def\PYG@tc##1{\textcolor[rgb]{0.19,0.19,0.19}{##1}}}
\def\PYG@tok@cp{\def\PYG@tc##1{\textcolor[rgb]{0.00,0.44,0.13}{##1}}}
\def\PYG@tok@gi{\def\PYG@tc##1{\textcolor[rgb]{0.00,0.63,0.00}{##1}}}
\def\PYG@tok@gh{\let\PYG@bf=\textbf\def\PYG@tc##1{\textcolor[rgb]{0.00,0.00,0.50}{##1}}}
\def\PYG@tok@ni{\let\PYG@bf=\textbf\def\PYG@tc##1{\textcolor[rgb]{0.84,0.33,0.22}{##1}}}
\def\PYG@tok@nl{\let\PYG@bf=\textbf\def\PYG@tc##1{\textcolor[rgb]{0.00,0.13,0.44}{##1}}}
\def\PYG@tok@nn{\let\PYG@bf=\textbf\def\PYG@tc##1{\textcolor[rgb]{0.05,0.52,0.71}{##1}}}
\def\PYG@tok@no{\def\PYG@tc##1{\textcolor[rgb]{0.38,0.68,0.84}{##1}}}
\def\PYG@tok@na{\def\PYG@tc##1{\textcolor[rgb]{0.25,0.44,0.63}{##1}}}
\def\PYG@tok@nb{\def\PYG@tc##1{\textcolor[rgb]{0.00,0.44,0.13}{##1}}}
\def\PYG@tok@nc{\let\PYG@bf=\textbf\def\PYG@tc##1{\textcolor[rgb]{0.05,0.52,0.71}{##1}}}
\def\PYG@tok@nd{\let\PYG@bf=\textbf\def\PYG@tc##1{\textcolor[rgb]{0.33,0.33,0.33}{##1}}}
\def\PYG@tok@ne{\def\PYG@tc##1{\textcolor[rgb]{0.00,0.44,0.13}{##1}}}
\def\PYG@tok@nf{\def\PYG@tc##1{\textcolor[rgb]{0.02,0.16,0.49}{##1}}}
\def\PYG@tok@si{\let\PYG@it=\textit\def\PYG@tc##1{\textcolor[rgb]{0.44,0.63,0.82}{##1}}}
\def\PYG@tok@s2{\def\PYG@tc##1{\textcolor[rgb]{0.25,0.44,0.63}{##1}}}
\def\PYG@tok@vi{\def\PYG@tc##1{\textcolor[rgb]{0.73,0.38,0.84}{##1}}}
\def\PYG@tok@nt{\let\PYG@bf=\textbf\def\PYG@tc##1{\textcolor[rgb]{0.02,0.16,0.45}{##1}}}
\def\PYG@tok@nv{\def\PYG@tc##1{\textcolor[rgb]{0.73,0.38,0.84}{##1}}}
\def\PYG@tok@s1{\def\PYG@tc##1{\textcolor[rgb]{0.25,0.44,0.63}{##1}}}
\def\PYG@tok@gp{\let\PYG@bf=\textbf\def\PYG@tc##1{\textcolor[rgb]{0.78,0.36,0.04}{##1}}}
\def\PYG@tok@sh{\def\PYG@tc##1{\textcolor[rgb]{0.25,0.44,0.63}{##1}}}
\def\PYG@tok@ow{\let\PYG@bf=\textbf\def\PYG@tc##1{\textcolor[rgb]{0.00,0.44,0.13}{##1}}}
\def\PYG@tok@sx{\def\PYG@tc##1{\textcolor[rgb]{0.78,0.36,0.04}{##1}}}
\def\PYG@tok@bp{\def\PYG@tc##1{\textcolor[rgb]{0.00,0.44,0.13}{##1}}}
\def\PYG@tok@c1{\let\PYG@it=\textit\def\PYG@tc##1{\textcolor[rgb]{0.25,0.50,0.56}{##1}}}
\def\PYG@tok@kc{\let\PYG@bf=\textbf\def\PYG@tc##1{\textcolor[rgb]{0.00,0.44,0.13}{##1}}}
\def\PYG@tok@c{\let\PYG@it=\textit\def\PYG@tc##1{\textcolor[rgb]{0.25,0.50,0.56}{##1}}}
\def\PYG@tok@mf{\def\PYG@tc##1{\textcolor[rgb]{0.13,0.50,0.31}{##1}}}
\def\PYG@tok@err{\def\PYG@bc##1{\fcolorbox[rgb]{1.00,0.00,0.00}{1,1,1}{##1}}}
\def\PYG@tok@kd{\let\PYG@bf=\textbf\def\PYG@tc##1{\textcolor[rgb]{0.00,0.44,0.13}{##1}}}
\def\PYG@tok@ss{\def\PYG@tc##1{\textcolor[rgb]{0.32,0.47,0.09}{##1}}}
\def\PYG@tok@sr{\def\PYG@tc##1{\textcolor[rgb]{0.14,0.33,0.53}{##1}}}
\def\PYG@tok@mo{\def\PYG@tc##1{\textcolor[rgb]{0.13,0.50,0.31}{##1}}}
\def\PYG@tok@mi{\def\PYG@tc##1{\textcolor[rgb]{0.13,0.50,0.31}{##1}}}
\def\PYG@tok@kn{\let\PYG@bf=\textbf\def\PYG@tc##1{\textcolor[rgb]{0.00,0.44,0.13}{##1}}}
\def\PYG@tok@o{\def\PYG@tc##1{\textcolor[rgb]{0.40,0.40,0.40}{##1}}}
\def\PYG@tok@kr{\let\PYG@bf=\textbf\def\PYG@tc##1{\textcolor[rgb]{0.00,0.44,0.13}{##1}}}
\def\PYG@tok@s{\def\PYG@tc##1{\textcolor[rgb]{0.25,0.44,0.63}{##1}}}
\def\PYG@tok@kp{\def\PYG@tc##1{\textcolor[rgb]{0.00,0.44,0.13}{##1}}}
\def\PYG@tok@w{\def\PYG@tc##1{\textcolor[rgb]{0.73,0.73,0.73}{##1}}}
\def\PYG@tok@kt{\def\PYG@tc##1{\textcolor[rgb]{0.56,0.13,0.00}{##1}}}
\def\PYG@tok@sc{\def\PYG@tc##1{\textcolor[rgb]{0.25,0.44,0.63}{##1}}}
\def\PYG@tok@sb{\def\PYG@tc##1{\textcolor[rgb]{0.25,0.44,0.63}{##1}}}
\def\PYG@tok@k{\let\PYG@bf=\textbf\def\PYG@tc##1{\textcolor[rgb]{0.00,0.44,0.13}{##1}}}
\def\PYG@tok@se{\let\PYG@bf=\textbf\def\PYG@tc##1{\textcolor[rgb]{0.25,0.44,0.63}{##1}}}
\def\PYG@tok@sd{\let\PYG@it=\textit\def\PYG@tc##1{\textcolor[rgb]{0.25,0.44,0.63}{##1}}}

\def\PYGZbs{\char`\\}
\def\PYGZus{\char`\_}
\def\PYGZob{\char`\{}
\def\PYGZcb{\char`\}}
\def\PYGZca{\char`\^}
% for compatibility with earlier versions
\def\PYGZat{@}
\def\PYGZlb{[}
\def\PYGZrb{]}
\makeatother

\begin{document}

\maketitle
\tableofcontents
\hypertarget{--doc-index}{}


Welcome to Dina Project manuals. Here you can learn more about Dina project, usage and development.

Contents:

\resetcurrentobjects
\hypertarget{--doc-users/index}{}

\chapter{User Manual}

Welcome

\resetcurrentobjects
\hypertarget{--doc-devel/index}{}

\chapter{Developers Manual}

We assume that you have enough experience in Python and Django before reading this section.

\resetcurrentobjects
\hypertarget{--doc-devel/policy}{}

\section{Dina-Project Development Policy}

This document contain the mercurial repository structure and specific commit policies.
If you have any idea please contact me at \textless{}\href{mailto:lxsameer@gnu.org}{lxsameer@gnu.org}\textgreater{}. if you find any bug in this document please file a bug in bug tracking system.
Since now i call Dina-Project directory (root directory) ``/'' so / sign refer to dina-project directory .
Please take a look at 2.2.1 section for definition of external applications.


\subsection{1. General Policy:}

At Dina Developer Community we warmly welcome to those people who wants to join us. so
as a Dina developers there is some notes that you should bear in mind:
\begin{itemize}
\item {} 
Respect to others not matter who you are.

\item {} 
if some one make a mistake in the code call him at private and tell him kindly.

\item {} 
Do not humiliate any one.

\item {} 
( think more ).

\end{itemize}


\subsection{2. Mercurial Repository Structure :}

There is some important issue that you should be aware of as Dina developer.


\subsubsection{2.1 Media files}

For a clean job its good to put just project itself under the dina-project repository,
but in addition to project code we have some files that play important roll in project
but are not the member of project itself for example , pictures and  are two important
part of framework/CMS view . its not a good idea to put the image source files (GIMP {\color{red}\bfseries{}*}.xcf
or Photoshop {\color{red}\bfseries{}*}.psd) in the project tree . we just use the image output not the source.
but the source file is needed because of our license, and we don't have any other
repository for this purpose now. ``So what should i do?''. please build a sub-directory
under you template folder with the ``media\_source'' and put you source file in there.
\begin{quote}

we will build a new repository and move those file ( because we don't want to increase
the size of Dina package ) to new repository.
\end{quote}


\subsubsection{2.2 Applications}

As decided in the Dina Developers Community, we will remove all external application
from Dina main repository due to reach these goals:
\begin{itemize}
\item {} 
Have a clean source tree

\item {} 
Minimalist package size

\item {} 
Separating Dina main code from applications code

\item {} 
Separating Dina task and bugs from applications one

\item {} 
and etc.

\end{itemize}


\paragraph{2.2.1 What are external applications ?}

\begin{notice}{warning}{Warning:}
define what exactly external application is
\end{notice}

But as said in 1.1 we don't have any other repository, so we have to put unnecessary apps
in the /apps directory.


\subsubsection{2.3 Branches}

We use default branch of mercurial repository as our unstable code repository. at each
release we will build a branch ant name it as the release version. also we have a branch
call ``Stable'' for our last stable code.


\subsubsection{2.4 Tags}

We use tag for special point of development. The code should tagged with its developer
GPG key with the meaningful message. for example important states of application of Dina
itself should tagged with GPG key of their last developer.


\subsection{3. Developing Policies:}

For gaining better and better team work ability, we need to follow these rulse:
\begin{itemize}
\item {} 
Dina Project release under the term of GPL version 2, so we have to put out license

\end{itemize}

header at the beginning of our source files. the source header should be exactly like:

\begin{Verbatim}[commandchars=\\\{\}]
\PYG{c}{\# ----------------------------------------------------------------------------}
\PYG{c}{\#    Dina Project}
\PYG{c}{\#    Copyright(C) 2010  Dina Project Developer Community}
\PYG{c}{\#}
\PYG{c}{\#    This program is free software; you can redistribute it and/or modify}
\PYG{c}{\#    it under the terms of the GNU General Public License as published by}
\PYG{c}{\#    the Free Software Foundation; either version 2 of the License, or}
\PYG{c}{\#    (at your option) any later version.}
\PYG{c}{\#}
\PYG{c}{\#    This program is distributed in the hope that it will be useful,}
\PYG{c}{\#    but WITHOUT ANY WARRANTY; without even the implied warranty of}
\PYG{c}{\#    MERCHANTABILITY or FITNESS FOR A PARTICULAR PURPOSE.  See the}
\PYG{c}{\#    GNU General Public License for more details.}
\PYG{c}{\#}
\PYG{c}{\#    You should have received a copy of the GNU General Public License along}
\PYG{c}{\#    with this program; if not, write to the Free Software Foundation, Inc.,}
\PYG{c}{\#    51 Franklin Street, Fifth Floor, Boston, MA 02110-1301 USA.}
\PYG{c}{\# -----------------------------------------------------------------------------}
\end{Verbatim}

please do not add your name in the license header as the author. if you want to other
developers know that your are the author of this peace of code or this section please
comment your name and contact a line or two after the license header.
e.g:

\begin{Verbatim}[commandchars=\\\{\}]
\PYG{c}{\# Sameer Rahmani \textless{}lxsameer@gnu.org\textgreater{}}
\end{Verbatim}
\begin{itemize}
\item {} 
If you use an external module that make Dina depend on in please ask the Dina Developers mailing list before that. So Dina Policy Committee can decide about the external code. We should not use external module or tools that are ncompatible with  GPL licenses.

\item {} 
If you use a peace of borrowed code in your code your should specify the source and license of borrowed code, otherwise your code will be removed.

\item {} 
If you have to use an external ( borrowed code ) within your source tree, please create a ``COPYING'' file and fill that with the external code license, and tell Dina Policy Committee about it.  !!! DO NOT USE GPL UNCOMPATIBLE CODE !!!

\item {} 
Comment your code as much as possible, so that every other developer can easily understand what was you doing.

\item {} 
Please READ PEP8 \textless{}\href{http://www.python.org/dev/peps/pep-0008/}{http://www.python.org/dev/peps/pep-0008/}\textgreater{} and USE it as your main coding style in Dina Project.

\item {} 
If you work on a new feature of new section please make a specification document for your work and put that document in /doc/devel/spec/ . Also make a user manual for your work and put that in /doc/user/ .

\end{itemize}

\begin{notice}{note}{Note:}
if you work on a  application put your user and spec documents under: APPDIR/doc/\{devel\textbar{}user\}/
\end{notice}
\begin{itemize}
\item {} 
If your looking for task to do, please take a look at task manager and bug manager system and pick a task that you want if the task or bug not assigned to any one else before. if it assigned before and you still want to work on it please contact to its author.

\item {} 
File a bug when you find any and work on that. so other developers can find out about that bug. After finishing your work close the bug.

\item {} 
If you made some changes in the code that other one works on it right now please contact the current author and tell him/her about your changes.

\item {} 
The commit strings should be meaningful, and show what's changed in code. Don't enter lazy strings as commit message please.

\item {} 
Broken or incomplete code should not committed.

\item {} 
Never ever merge the branches with each other till you know what you do. After merging branches please tell every one about merging.

\item {} 
Always add new code to default branch and never work on other branch unless you know what you do

\item {} 
Never split the project heads until you know what you do.

\item {} 
Never use -f option for `hg push' ( -f means force push )

\item {} 
Always choose meaningful name for your code components.

\end{itemize}

\resetcurrentobjects
\hypertarget{--doc-devel/spec/index}{}

\section{Dina features specifications}

Each of Dina features that implemened in the source code, have a specification here that explains the mechanism
and internals of the part, and how to use that feature in your code.

\resetcurrentobjects
\hypertarget{--doc-devel/spec/layout}{}

\subsection{Layout Manager}

Dina have a powerful and flexible mechanism that provide dynamic template rendering. By using
\code{Layout Manager} you can define your own section in the template code easily. for example
\code{top} or \code{side} section and then choose to render which components in which section.

Dina provide a simple to use UI to fill the template section with template components such as
\code{news} , \code{blog} , \code{menu system} and what ever your installed in you Dina package.


\subsubsection{How does Layout Manager works? --- Concepts}

Dina use a collection of tools and modules in order to manage the front-end layouts as a unigue
system that is called \code{Layout manager}. Layout manager is not an application or an isolated
python package by itself, it use some other parts of Dina such as template loader template
cache system and etc to manage dynamic template rendering.

\resetcurrentobjects
\hypertarget{--doc-devel/spec/logging}{}

\subsection{Loggin System specification}

Dina use logging system to provide useful informations at the runtime that allow better
and faster debugging. Also privent developers to use \emph{print} statement in there code
that could break Dina code under WSGI and mod\_python.


\subsubsection{How does Logging system works ?}

Logging subsystem code live under \code{dina/log/\_\_init\_\_.py} at this time. It lookup for
the logging configuration in settings.py and provide a class called \hyperlink{Logger}{\code{Logger}}. Each
part of Dina that need logging system should build an instance from the Logger class,
and use the instance for produce some logs. Dina initialize its logging system configuration
in the settings.py file. So each time that web server runs the settings.py script logging
configuration initialized once. for more detail take a look at code comments.


\subsubsection{How to use logging system ?}

Using the logger system is so simple, you can make an instance from the \hyperlink{Logger}{\code{Logger}} class
and use that to produce logs. Here is the explaination for \hyperlink{Logger}{\code{Logger}} class.

\begin{notice}{note}{Note:}
Parameter names are just for better understanding and not exactly the same as real ones.
\end{notice}
\index{Logger (built-in class)}

\hypertarget{Logger}{}\begin{classdesc}{Logger}{logger\_name}\end{classdesc}

Logger class use \code{logger\_name} as the name of log producer, for example:

\begin{Verbatim}[commandchars=@\[\]]
@PYGZlb[]2010-08-25 11:54:42@PYGZrb[] @PYGZlb[]Template cache class@PYGZrb[], line:64-@textgreater[] DEBUG    : "TemplateQueryCache class inti."
\end{Verbatim}

The above log snippet taken from Dina log. You can see that \code{Template cache class} procude a \code{DEBUG}
log in line \code{64} of its module. In this example \code{Template cache class} used as \code{logger\_name} parameter
for \hyperlink{Logger}{\code{Logger}} class.
\index{debug() (Logger method)}

\hypertarget{Logger.debug}{}\begin{methoddesc}[Logger]{debug}{log\_string}\end{methoddesc}

This method produce a log with the priority level of DEBUG and \code{log\_string} data.
\index{info() (Logger method)}

\hypertarget{Logger.info}{}\begin{methoddesc}[Logger]{info}{log\_string}\end{methoddesc}

This method produce a log with the priority level of INFO and \code{log\_string} data.
\index{warning() (Logger method)}

\hypertarget{Logger.warning}{}\begin{methoddesc}[Logger]{warning}{log\_string}\end{methoddesc}

This method produce a log with the priority level of WARNING and \code{log\_string} data.
\index{error() (Logger method)}

\hypertarget{Logger.error}{}\begin{methoddesc}[Logger]{error}{log\_string}\end{methoddesc}

This method produce a log with the priority level of ERROR and \code{log\_string} data.
\index{critical() (Logger method)}

\hypertarget{Logger.critical}{}\begin{methoddesc}[Logger]{critical}{log\_string}\end{methoddesc}

This method produce a log with the priority level of CRITICAL and \code{log\_string} data.

A simple example:

\begin{Verbatim}[commandchars=\\\{\}]
\PYG{k+kn}{from} \PYG{n+nn}{dina.log} \PYG{k+kn}{import} \PYG{n}{Logger}

\PYG{n}{logger} \PYG{o}{=} \PYG{n}{Logger} \PYG{p}{(}\PYG{l+s}{"}\PYG{l+s}{test module}\PYG{l+s}{"}\PYG{p}{)}
\PYG{n}{logger}\PYG{o}{.}\PYG{n}{info} \PYG{p}{(}\PYG{l+s}{"}\PYG{l+s}{Some log}\PYG{l+s}{"}\PYG{p}{)}
\end{Verbatim}

This code snippet will produce a log entry like:

\begin{Verbatim}[commandchars=@\[\]]
@PYGZlb[]2010-08-25 11:54:42@PYGZrb[] @PYGZlb[]test module@PYGZrb[], line:4-@textgreater[] INFO    : "Some log"
\end{Verbatim}


\strong{See Also:}


Please read the official python documentation for logging module, current logging system use the python logging module.




\chapter{Indices and tables}
\begin{itemize}
\item {} 
\emph{Index}

\item {} 
\emph{Module Index}

\item {} 
\emph{Search Page}

\end{itemize}


\renewcommand{\indexname}{Module Index}
\printmodindex
\renewcommand{\indexname}{Index}
\printindex
\end{document}
