\input texinfo   @c -*-texinfo-*-
@c %**start of header
@setfilename Debbox_Development_Reference.tex
@settitle Debbox Development Reference 1.0
@c %**end of header
@copying
Debbox development reference, version 1.0.

Copyright @copyright{} 2011 Some Hackers In Town street computing group.
@end copying

@titlepage
@title Debbox Development Reference
@subtitle Version 1.0
@author By:
@author Sameer Rahmani

@c The following two commands start the copyright page.
@page
@vskip 0pt plus 1filll
@insertcopying
@page
Copyright (C)  2011  Some Hackers In Town street computing group.
Permission is granted to copy, distribute and/or modify this document
under the terms of the GNU Free Documentation License, Version 1.3
or any later version published by the Free Software Foundation;
with no Invariant Sections, no Front-Cover Texts, and no Back-Cover Texts.
A copy of the license is included in the section entitled "GNU
Free Documentation License".
@end titlepage

@c Output the table of contents at the beginning.
@contents
@menu
* Logging::     Logging system for debbox
* Index::            Complete index.
@end menu

@node Logging
@chapter Logging System
Debbox use the original python logging facility, but with some configuration and changes. logging configuration live at @file{settings.py}
You should use logging facility in you code, it is very important to collect information about current process, Log outputs are very useful and
valuable debug time informations.

@section Logging configuration
There is no need to configuring Debbox logger by default. In any case you can configure logger by changing its configuration variables in 
@file{settings.py}. Here is the list of logger configurations:
@sp 1
@itemize
@item
@strong{@code{VERBOSE}}: This option allow you to choose a level for Debbox Logger. for example if you set this option to @strong{WARNING}, only logs 
higher than @strong{WARNING} level will be appear in logger output. default value of this option is @strong{0}, so logger ouput will contains all the logs in all logger levels.

@item
@strong{@code{LOG_FORMAT}}: This option indicate the format of logger output string.
@item
@strong{@code{LOG_MAX_BYTES}}: Numbers of bytes that will put in the Debbox log files. If logger reache this limit, it will create an archive from current log file in the same directory.
@item
@strong{@code{LOG_BACKUP_COUNT}}: Numbers of backups for log file. also you can think of it as current log file archives.
@strong{@code{LOG_FILENAME}}: Where to store log recordes.
@end itemize
@sp 1
@center
@quotation Note
For the list of complete options of logger take a look at @file{settings.py}.
@end quotation
@sp 1

@section Logger usage
For using Debbox logger all you need it to import it and use it. for example:
@sp 1
@example
@code{from core.log import logger}


@code{def someview(request):}
@code{    logger.debug("Hello logger")}
@end example
@sp 1
Above example will produce:
@example
@code{[2011-01-22 05:50:02] [views.py-someview], line:5-> DEBUG     : "Hello logger"}
@end example
@sp1
As you can see in Debbox logger record, default configuration of Debbox logger have some interesting information in the log record like where log produced (in this example views.py module at someview function in line 5) and the time of record.
@sp1
@quotation See Also
For more information about logger take a look at official python logger documentation.
@end quotation
@node Index
@unnumbered Index

@printindex cp

@bye
