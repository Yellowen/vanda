\input texinfo   @c -*-texinfo-*-
@c %**start of header
@setfilename Debbox_Development.tex
@settitle Debbox Development Guide 1.0
@c %**end of header
@copying
Debbox development Guide, version 1.0.

Copyright @copyright{} 2011 Some Hackers In Town street computing group.
@end copying

@titlepage
@title Debbox Development Guide
@subtitle Version 1.0
@author By:
@author Sameer Rahmani

@c The following two commands start the copyright page.
@page
@vskip 0pt plus 1filll
@insertcopying
@page
Copyright (C)  2011  Some Hackers In Town street computing group.
Permission is granted to copy, distribute and/or modify this document
under the terms of the GNU Free Documentation License, Version 1.3
or any later version published by the Free Software Foundation;
with no Invariant Sections, no Front-Cover Texts, and no Back-Cover Texts.
A copy of the license is included in the section entitled "GNU
Free Documentation License".
@end titlepage

@c Output the table of contents at the beginning.
@contents
@menu
* Requirements::        Requirements for developing Debbox
* Index::            Complete index.
@end menu
@c TODO: add an Introduction chapter that talk about this article
@node Requirements
@firstparagraphindent insert
@chapter Requirements
Before jumping in the Debbox development process make sure that you have enough knowledge to use
Python programming language and Django framework. These two are the most important requirements
for Debbox development.

@section Required packages
Debbox will use an isolated environment in the deployed state, to prevent unwanted changes in the
Host OS. Here is a list of required packages with their details for building a Debbox virtual environment:

@itemize
@item
@strong{python-virtualenv}: Debbox use this script to create e virtual environments.
@item
@strong{libev-dev}: This packages is required if you want to use FAPWS3 as the web server backed for Debbox.
@item
@strong{libevent-1.4.2}: This packages is required if you want to use GEvent as the web server backed for Debbox.
@item
@strong{libevent-dev}: This packages is required if you want to use GEvent as the web server backed for Debbox.
@end itemize

@section Building virtual environment
After installing required packages, you can easily build a environment using @file{debbox/bin/envcreator.sh} script. @file{envcreator.sh}
will build a environment in the current working directory with the name of @emph{env}, and install Django, FAPWS3, GEvent and PIL in created
virtualenv directory.

@sp 1
@quotation
@strong{Note}: if you already have a @file{env} directory @file{envcreator.sh} will not clean that environment just installs new packages. But 
if you want to clean the exists environment up, just use @emph{--clean} parameter with @file{envcreator.sh} .
@end quotation
@sp 1
@quotation
@strong{Note}: @file{envcreator.sh} script will install both of FAPWS3 and GEvent in virtual environment. Debbox will use one of them at a time
so if you don't want tu install one of them, just comment the corresponding code in @file{envcreator.sh}  
@end quotation
@sp 1
@quotation
@strong{Warning}: Do not commit you comments (see above note) in @file{envcreator.sh} on the main Repository.
@end quotation
@sp 1
@node Index
@unnumbered Index

@printindex cp

@bye
