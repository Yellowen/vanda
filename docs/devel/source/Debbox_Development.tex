\input texinfo   @c -*-texinfo-*-
@c %**start of header
@setfilename Debbox_Development.tex
@settitle Debbox Development Guide 1.0
@c %**end of header
@copying
Debbox development Guide, version 1.0.

Copyright @copyright{} 2011 Some Hackers In Town street computing group.
@end copying

@titlepage
@title Debbox Development Guide
@subtitle Version 1.0
@author By:
@author Sameer Rahmani

@c The following two commands start the copyright page.
@page
@vskip 0pt plus 1filll
@insertcopying
@page
Copyright (C)  2011  Some Hackers In Town street computing group.
Permission is granted to copy, distribute and/or modify this document
under the terms of the GNU Free Documentation License, Version 1.3
or any later version published by the Free Software Foundation;
with no Invariant Sections, no Front-Cover Texts, and no Back-Cover Texts.
A copy of the license is included in the section entitled "GNU
Free Documentation License".
@end titlepage

@c Output the table of contents at the beginning.
@contents
@menu
* Development Policies::        The policies that each developer should be aware of.
* Requirements::        Requirements for developing Debbox.
* Index::            Complete index.
@end menu
@c TODO: add an Introduction chapter that talk about this article
@node Development Policies
@chapter Development Policies
@c TODO: use the email command to make lxsameer@gnu.org an email
In this chapter you will read the most important policies for developing Debbox. If you have any idea please contact me at @email{lxsameer@@gnu.org}.
@section General Policies    
At Debbox Developers Community we warmly welcome those people who wants to join us. so
as a Debbox developer there is some notes that you should bear in mind:
@sp 1
@itemize
@item
Respect to others not matter who you are.
@item
If some one make a mistake in the code call him at private and tell him kindly.
@item
Do not humiliate any one.
@end itemize

@section Developing Policies
For gaining better and better team work ability, we need to follow these rulse:
@sp 1
@itemize
@item
Debbox Project release under the term of GPL version 2, so we have to put out license
header at the beginning of our source files. take a look at a python file to find out how to use 
license header.

please do not add your name in the license header as the author. if you want to other developers know that your are
the author of this piece of code or this section please comment your name and contact in a line or two after the license header. 
e.g
@sp 1
@quotation
@code{# Sameer Rahmani <lxsameer@@gnu.org>}
@end quotation
@sp 1
@item
If you use an external module that make Debbox depend on that please ask in the Dina Developers mailing list before using that.
So Debbox Policy Committee can decide about the external code. We should not use external module or tools that are ncompatible with  GPL licenses.
@item
If you use a piece of borrowed code in your code, you should specify the source and license of borrowed code, otherwise your code will be removed. 
@item
If you have to use an external (borrowed code) within Debbox source tree, please create a "COPYING" file and fill that with the external code license after get the agreement of Debbox Policy Committee .  @strong{!!! DO NOT USE GPL INCOMPATIBLE CODE !!!}
@item
Comment your code as much as possible, so that every developer can easily understand what did you do.
@item
Please read PEP8 @url{http://www.python.org/dev/peps/pep-0008/} and @strong{USE} it as your main coding style in Debbox Project.
@sp 1
@quotation Note
You can use some tools like @file{pep8} for check your coding style. Also some editors like Emacs can easily check your coding style while 
editing your code.
@end quotation
@sp 1
@item
If you made some changes in the code that other one works on it right now please contact the current author and tell him/her about your changes.
@item
The commit strings should be meaningful, and show what's changed in code. Don't enter lazy strings as commit message please.
@item
Do not commit broken or incomplete code.
@item
Never ever merge the branches with each other till you know what you're doing. After merging branches please tell every one about merging.
@end itemize




@node Requirements
@firstparagraphindent insert
@chapter Requirements
Before jumping in the Debbox development process make sure that you have enough knowledge to use
Python programming language and Django framework. These two are the most important requirements
for Debbox development.

@section Required packages
Debbox will use an isolated environment in the deployed state, to prevent unwanted changes in the
Host OS. Here is a list of required packages with their details for building a Debbox virtual environment:
@sp 1
@itemize
@item
@strong{python-virtualenv}: Debbox use this script to create e virtual environments.
@item
@strong{libev-dev}: This packages is required if you want to use FAPWS3 as the web server backed for Debbox.
@item
@strong{libevent-1.4.2}: This packages is required if you want to use GEvent as the web server backed for Debbox.
@item
@strong{libevent-dev}: This packages is required if you want to use GEvent as the web server backed for Debbox.
@end itemize

@section Building virtual environment
After installing required packages, you can easily build a environment using @file{debbox/bin/envcreator.sh} script. @file{envcreator.sh}
will build a environment in the current working directory with the name of @emph{env}, and install Django, FAPWS3, GEvent and PIL in created
virtualenv directory.

@sp 1
@quotation
@strong{Note}: if you already have a @file{env} directory @file{envcreator.sh} will not clean that environment just installs new packages. But 
if you want to clean the exists environment up, just use @emph{--clean} parameter with @file{envcreator.sh} .
@end quotation
@sp 1
@quotation
@strong{Note}: @file{envcreator.sh} script will install both of FAPWS3 and GEvent in virtual environment. Debbox will use one of them at a time
so if you don't want tu install one of them, just comment the corresponding code in @file{envcreator.sh}  
@end quotation
@sp 1
@quotation
@strong{Warning}: Do not commit you comments (see above note) in @file{envcreator.sh} on the main Repository.
@end quotation
@sp 1
Virtual environment will simulate a naked *nix environment with minimal dependencies installed. To activating provided virtualenv you can do like thise
@sp 1
@command{$ . env/bin/activate}
@sp 1
and for exit the virtual environment use @command{deactivate} command.

@section Running Debbox server
By installing requirement and building a virtual environment, now its time to run Web server of Debbox. There is two web server backend for Debbox, 
FAPWS3 and GEvent. both of them are fastest web servers in all ages. Sine there is some fundamental difference between those two Debbox allow user
to choose one of them. GEvent is the default backend. for running Debbox web server you can use the @command{server.py} command after switching to 
virtual environment like:
@sp 1
@command{$ python server.py}
@sp 1
The above command run the default backend on the @emph{localhost:8000}. If you want to change the backend use @option{--backend} option just like:
@sp 1
@command{$ python server.py --backend=fapws3}
@sp 1
for more information about @command{server.py} take a look at its help output.
@node Index
@unnumbered Index

@printindex cp

@bye
