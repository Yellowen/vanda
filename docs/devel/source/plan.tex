\input texinfo   @c -*-texinfo-*-
@c %**start of header
@setfilename Plan.tex
@settitle Debbox Development Plan 1.0
@c %**end of header
@copying
Debbox development plan document, version 1.0.

Copyright @copyright{} 2011 Some Hackers In Town street computing group.
@end copying
@titlepage
@title Debbox Development Plan
@subtitle Version 1.0
@author By:
@author Sameer Rahmani

@c The following two commands start the copyright page.
@page
@vskip 0pt plus 1filll
@insertcopying
@end titlepage

@c Output the table of contents at the beginning.
@contents

@menu
* Debbox idea::    The ideas behind Debbox.
* Requirements::   The requirements for building Debbox V0

@end menu

@c TODO: add an Introduction chapter that talk about this article
@node Debbox idea
@firstparagraphindent insert
@chapter Debbox idea
There is just a few administration web application around and most of them is dead now (not developed any more), and those
which is active have their own issues, for example:
@itemize
@item
Do not following popular standards like LBS.
@item
Stick with just some sort of software like Apache web server
@item
Don't have a user friendly UI
@item
Do not support multiple server configuration
@item 
Configuring server in a nasty way.
@item
and etc
@end itemize

Also system administrators limited to softwares which their administration panel provide them.
Most of the time there is no choice, for example most of the current panels user Apache as their
default web server and did not support other web servers like lighttpd or nginx.

The main idea behind Debbox is to build a modern administration panel for Debian GNU/Linux operating 
system. Building for Debian GNU/Linux did not mean that Debbox should work on Debian, Debbox should
follow Debian provided standards like FHS (Filesystem Hierarchy standard), python-policy or php-policy
and so on. one of the weakness of other panel is that administrator will confuse by using them for example
Kloxo configure apache web server in an strange way and those administrator who use kloxo for the first time
should search for a user virtualhost file in GNU/Linux root partition to change that by hand at that is a pain 
in the ass problem. So by following Debian standards, Debbox will be Admin friendly panel. 

@c add more idea about debbox       

@node Requirements
@chapter Requirements

After understanding Debbox idea, now it's time to ask this question: What does Debbox need for a 'Just working' version?


@node Index
@unnumbered Index

@printindex cp

@bye
