\input texinfo   @c -*-texinfo-*-
@c %**start of header
@setfilename Plan.tex
@settitle Debbox Development Plan 1.0
@c %**end of header
@copying
Debbox development plan document, version 1.0.
Copyright @copyright{} 2011 Some Hackers In Town street computing group.
@end copying

@titlepage
@title Debbox Development Plan
@subtitle Version 1.0
@author By:
@author Sameer Rahmani

@c The following two commands start the copyright page.
@page
@vskip 0pt plus 1filll
@insertcopying

@page
This document belongs to Debbox project and discuss the Development plan for version 0 of Debbox. if you're not 
a Debbox developer this document is not for you.

@end titlepage

@c Output the table of contents at the beginning.
@contents


@c TODO: add an Introduction chapter that talk about this article
@node What is Debbox?
@firstparagraphindent insert
@chapter What is Debbox?
Debbox is an administration panel only for Debian GNU/Linux, that provide an easy to use UI for Admin user and normal users. 

@section What is the advantage of Debbox?
There is just a few administration web application around and most of them is dead now (not developed any more), and those
which is active have their own issues, for example:
@itemize
@item
Do not following popular standards like LBS.
@item
Stick with just some sort of software like Apache web server
@item
Don't have a user friendly UI
@item
Do not support multiple server configuration
@item 
Configuring server in a nasty way.
@item
and etc
@end itemize

Also system administrators limited to softwares which their administration panel provide them.
Most of the time there is no choice, for example most of the current panels user Apache as their
default web server and did not support other web servers like lighttpd or nginx.

The main idea behind Debbox is to build a modern administration panel for Debian GNU/Linux operating 
system. Building for Debian GNU/Linux did not mean that Debbox should work on Debian, Debbox should
follow Debian provided standards like FHS (Filesystem Hierarchy standard), python-policy or php-policy
and so on. one of the weakness of other panel is that administrator will confuse by using them for example
Kloxo configure apache web server in an strange way and those administrator who use kloxo for the first time
should search for a user virtualhost file in GNU/Linux root partition to change that by hand at that is a pain 
in the ass problem. So by following Debian standards, Debbox will be Admin friendly panel. 

Most of other panels did not respect to the level of user knowledge, and design the UI as they pleased. Because of that
some normal users with low computer knowledge will not feel comfortable with the UI. but Debbox load the UI with the details
 that user will understand probably, and let user to specify his/her level.

Those reasons that mentioned above are just some of many problem that my friends and i face to them in our career. So
we decided to build a new Web application that fit our needs.


@c add more idea about debbox       

@node Planning
@chapter Planning
The important task in creating a software product is extracting the requirements or requirements analysis. 
@section Requirements
After analyzing other panels weaknesses and advantages we generate a list of requirement for Debbox as follow:
@c START OF LIST
@itemize
@c SOME EXAMPLE OF USING texinfo lists
@c to not edit this example (lxsameer) will remove them
@item
item 1
@item
item2
@itemize
@item 
subitem1
@end itemize
@item
item3
@enumerate 
@item
subitem 4
@end enumerate
@c ADD you list here -------------------
@c for example
@item
nazare behnam
@item
nazare mohamad
@c -------------------------------------
@end itemize
@c END OF LIST

@node Index
@unnumbered Index

@printindex cp

@bye
